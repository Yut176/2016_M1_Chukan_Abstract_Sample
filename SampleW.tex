\documentclass[twocolumn, 10ptj, dvipdfmx]{jsarticle}
% タイトル部分の微調整
\makeatletter
\def\@maketitle{%
  \newpage\null
  \vskip 1.08em
  \begin{center}%
    \let\footnote\thanks
    {\@title \par}%
    \vskip 1.36em
    {\large
      \lineskip .5em
      \begin{tabular}[t]{c}%
        \@author
      \end{tabular}\par}%
    % \vskip 1em
    % {\large \@date}%
  \end{center}%
  \par\vskip 0.92em
  \ifvoid\@abstractbox\else\centerline{\box\@abstractbox}\vskip1.5em\fi
}
\def\mojiparline#1{
    \newcounter{mpl}
    \setcounter{mpl}{#1}
    \@tempdima=\linewidth
    \advance\@tempdima by-\value{mpl}zw
    \addtocounter{mpl}{-1}
    \divide\@tempdima by \value{mpl}
    \advance\kanjiskip by\@tempdima
    \advance\parindent by\@tempdima
}
\makeatother
\def\linesparpage#1{
    \baselineskip=\textheight
    \divide\baselineskip by #1
}
\usepackage[dvipdfm]{geometry}
\geometry{top=20truemm, bottom=25truemm, left=20truemm, right=20truemm}
\pagestyle{empty}
% 以上設定
%%%%%%%%%%%%%%%%%%%%%%%%%%%%%%%%%%%%%%%%%%%%%%%%%%%%%%%%
% 使用パッケージ
%
% 色つける
\usepackage{xcolor}
% 文字幅を固定幅にする
\usepackage{pxmonja}
%%%%%%%%%%%%%%%%%%%%%%%%%%%%%%%%%%%%%%%%%%%%%%%%%%%%%%%%
% ここを編集
\newcommand{\TITLE}{日本語タイトル(フォントサイズは14pt)}
\newcommand{\ETITLE}{English title (Font size is 12pt)}
\newcommand{\NUMBER}{MF16000}
\newcommand{\NAME}{シス理\quad{}太郎}
\newcommand{\MAJOR}{○○○○○○○○研究}
\newcommand{\TEACHER}{芝浦\quad{}工太郎}
%%%%%%%%%%%%%%%%%%%%%%%%%%%%%%%%%%%%%%%%%%%%%%%%%%%%%%%%
% 設定
\makeatletter
\title{{\textbf{\fontsize{14truept}{3pt}\selectfont\TITLE\\[10.0truept]{}%
\fontsize{12truept}{\n@baseline}\selectfont\ETITLE}}}
\author{{\fontsize{12truept}{\n@baseline}\selectfont
\begin{tabular}{p{16.0em}rl}
  システム理工学専攻 & \NUMBER & \NAME \\
  \MAJOR & 指導教員(担当教員) & \TEACHER
\end{tabular}
}}
\makeatother
%%%%%%%%%%%%%%%%%%%%%%%%%%%%%%%%%%%%%%%%%%%%%%%%%%%%%%%%
\begin{document}
\maketitle
% 文字サイズ, 幅の調整
\fontsize{10truept}{15.4truept}\selectfont
\mojiparline{20}
%%%%%%%%%%%%%%%%%%%%%%%%%%%%%%%%%%%%%%%%%%%%%%%%%%%%%%%%
%%%%%%%%%%%%%%%%%%%%%%%%%%%%%%%%%%%%%%%%%%%%%%%%%%%%%%%%
%
% ここから本文
\noindent
これより本文.本文は2段組構成.\\
(2016年11月9日版).
\\[\Cvs]
以下に基本設定を示す.この雛形は基本設定に則っている.
これを目安に資料を作成のこと.
\quad{}LaTeX\hspace{0.4zw} 等を用いる場合もこの設定を参考に作成のこと.
\\[\Cvs]
= 基本設定 =\\
\hspace{-1.0zw}
\begin{tabular}{lp{13.5zw}}
  $<$フォント(目安)$>$ &\\
  フォント種類: & 全角は MS 明朝体,あるいは MS ゴシック体等.半角は Century,Times New Roman 等.\\
  フォントサイズ: & タイトルは 14pt(英文タイトル12pt).学籍番号,氏名等は 12pt.本文は 10pt.\\
  $<$文字数(目安)$>$ &\\
  文字数: & 1段1行あたり,20文字.\\
  行数: & 1ページ目,37行 \\
  & 2ページ目,45行 \\
  $<$余白$>$ &\\
  上,左,右: & 20mm \\
  下: & 25mm
\end{tabular}
%
\\[\Cvs]
$<$その他指定事項$>$
\vspace{-1zw}
\begin{itemize}
  \item 「日本語タイトル行」,「英語タイトル行」ともに,センタリング,かつ太文字.
  \item 「英語タイトル行」と「学籍番号・氏名行」との間は改行.
  \item 「学籍番号・氏名行」と「本文」との間は改行.
  \item 「学籍番号」は半角とし,アルファベットは大文字.
  \item 「氏名」は全角とし,姓と名の間は1スペース空ける.
\end{itemize}
\newpage~
\\
$<$概要の構成例$>$\\
\color{red}\bfseries
1.要約{\fontsize{8truept}{16}\selectfont (Abstract)}:英語{\fontsize{8truept}{16}\selectfont (English)(Font size: 8pt, Bold)
: 日本人入学学生は英語で書くこと.外国語入学学生は,可能ならば日本語で書くこと.}\\[0.2em]
\color{black}\mdseries%
2.研究背景, 意義, 研究目的など\\
3.本論 (方法,結果,考察)\\
4.まとめ,結論 (得られたこと,今後の課題)\\
参考文献 \#必ずつけること(省略表記でよい)
\\[\Cvs]
\color{red}\bfseries
※なお,日本人学生でもすべて英文で作成しても構わない.
\color{black}\mdseries
%
% 飽きたので終了
%
\end{document}
% EOF
